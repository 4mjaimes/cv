\documentclass[9pt]{developercv}

\begin{document}
	
	% ------------------------------------ Header ----------------------------------
	\begin{minipage}[t]{0.55\textwidth}
		\vspace{-\baselineskip}
		\colorbox{black}{{\HUGE\textcolor{white}{\textbf{\MakeUppercase{Manuel Jaimes}}}}}
	\end{minipage}
	\begin{minipage}[t]{0.275\textwidth}
		\vspace{-\baselineskip}
		\icon{Phone}{10}{\href{tel:573044777652}{(+57) 304 477 7652}}\\
		\icon{Envelope}{10}{\href{mailto:4mjaimes@gmail.com}{4mjaimes@gmail.com}}\\
		%    \icon{Globe}{12}{\href{https://pablotrinidad.me}{pablotrinidad.me}}\\
	\end{minipage}
	\begin{minipage}[t]{0.275\textwidth}
		\vspace{-\baselineskip}
		\icon{Github}{10}{\href{https://github.com/4mjaimes}{4mjaimes}}\\
		\icon{Linkedin}{10}{\href{https://www.linkedin.com/in/4mjaimes/}{4mjaimes}}\\
	\end{minipage}
	
	\cvsect{Skills}
	\begin{minipage}{0.6\textwidth}	
		\begin{itemize}[noitemsep,nolistsep,leftmargin=*]
			\item Capacidad para trabajar y comunicarse como parte de un equipo.
			\item Analizar y escribir sobre temas complejos de una manera clara y concisa.
			\item Comprender, analizar y satisfacer las necesidades tecnológicas y empresariales de clientes y compañeros de trabajo.
			\item Capacidad para adaptarse rápidamente y aprender nuevas tecnologías.
		\end{itemize}	
	\end{minipage}
	\begin{minipage}{0.4\textwidth}
		\begin{barchart}{5.5}
			\baritem{C\#}{80}
			\baritem{Entity Framework}{65}
			\baritem{JavaScript}{60}
			\baritem{.NET}{75}
			\baritem{React}{57}
		\end{barchart}
	\end{minipage}
	\vspace{-\baselineskip}
	% ---------------------------------- Experience --------------------------------
	\cvsect{Experiencia}
	\begin{entrylist}
		\entry
		{Actual\\Sep 2019}
		{Desarrollador .NET}
		{Grupo Digital}
		{
			Desarrollador en diversos proyectos siguiendo la metodología Scrum.\\
			Desarrollo de microservicios e integración frontend y backend.\\
			\texttt{.NET Core}\slashsep
			\texttt{React}\slashsep
			\texttt{Entity Framework Core}\slashsep
			\texttt{PostgreSQL}\slashsep
			\texttt{ASP.NET Identity}\slashsep
			\texttt{Redux}
		}
		\entry
		{Nov 2017\\Jun 2019}
		{Desarrollador}
		{Sistemas y Computadores- SYC}
		{
			Mantener y mejorar la aplicación de generación de informes usada por los diferentes proyectos de la compañía.\\
			\texttt{.NET Framework\slashsep JavaScript\slashsep ADO.NET\slashsep Bootstrap}\\
			Desarrollador principal del proyecto “Edesk inmobiliario - Camacol B\&C”. Diseño del esquema de base de datos para representar la información recopilada de la extracción de texto de diversos documentos. Escribir consultas y transacciones T-SQL.\\
			\texttt{.NET Framework\slashsep LINQ\slashsep SQL Server\slashsep Definición de producto}
		}
		\entry
		{Ago - Dic 2016\\Madrid, España}
		{Investigador practicante}
		{UAM - UC3M}
		{
			Investigador en el area de redes de computadoras en High Performance Computing and Networking Group y Laboratorio de Computación Ubicua.
			Mejoras al proceso de generación de suma de chequeo de los paquetes de red desarrollado durante la tesis de grado.
			Formación en el área de procesamiento de redes utilizando circuitos reprogramables FPGAs.
			Documentación de la plataforma para la comunidad NetFPGA Latinoamérica.\\
			\texttt{FPGA\slashsep VHDL\slashsep C\slashsep Linux\slashsep Documentación técnica}                
		}
	\end{entrylist}
	\vspace{-\baselineskip}
	
	% ---------------------------------- Education ---------------------------------
	\cvsect{Educación}
	\begin{entrylist}
		\entry
		{Jul 2010 - Dic 2016}
		{Ingeniero de sistemas}
		{Universidad Autónoma de Bucaramanga, UNAB}
		{	Énfasis en programación de dispositivos de red utilizando circuitos reprogramables FPGAs.\\
			Asistente de posgrados de la facultad de ingeniería de sistemas.
		}
		\entry
		{Ago - Nov 2019}
		{Escuela de JavaScript}
		{Platzi}
		{Educación online y presencial para especializarse en desarrollo Full Stack con JavaScript \textit{(MERN Stack)}. Formación compuesta por 10 cursos en los que se desarrolló un clon de Netflix.}		
	\end{entrylist}
	\vspace{-\baselineskip}
	
	\begin{minipage}{\textwidth}
		\cvsect{Actividad extracurricular}
		
		\begin{itemize}[noitemsep,nolistsep,leftmargin=*]
			\item \textbf{Coordinador,} Semillero de investigación en Telemática 2011 - 2016. Aporte a dos tesis de maestría y una de doctorado.
			\item \textbf{Investigador,} Convocatoria 617 de 2013 - Semilleros Colciencias. Proyecto: \textit{“Interfaz software/hardware para la comunicación de aplicaciones de red con la plataforma NetFPGA como soporte a procesos de investigación y docencia en el área de redes de computadoras”}.
			\item \textbf{NetFPGA Latinoamérica,} Cofundador. Comunidad académica orientada a la socialización y desarrollo de proyectos sobre la plataforma NetFPGA.
		\end{itemize}
	\end{minipage}

	\begin{minipage}{\textwidth}
	\cvsect{Publicaciones}
	
	NetFPGA: Docencia e Investigación para la Innovación en Redes de Computadoras Cesar D Guerrero, Manuel Jaimes, Yolanda Carreño, Antonio Lobo World Engineering Education Forum, 2013, Cartagena, Colombia
	\end{minipage}
	
	% -------------------------------- Additional info -----------------------------
	\begin{minipage}{0.4\textwidth}
		%\vspace{-\baselineskip}
		\cvsect{Reconocimientos}
		
		\begin{itemize}[noitemsep,nolistsep,leftmargin=*]
			\item \textbf{Tesis laureada}, UNAB
			\item \textbf{Excelencia en investigación}, UNAB
			\item \textbf{Becario}, Banco Santander
		\end{itemize}
	\end{minipage}
	\begin{minipage}{0.3\textwidth}
		%\vspace{-\baselineskip}
		\cvsect{Cursos}
		
		\inverseicon{Book}{10}{\href{https://platzi.com/@4mjaimes/}{Perfil Platzi \ExternalLink}}\\
		\inverseicon{Book}{10}{\href{https://app.pluralsight.com/profile/4mjaimes}{Perfil Pluralsight \ExternalLink}}
	\end{minipage}
	\begin{minipage}{0.3\textwidth}
		\cvsect{Idiomas}
		
		\textbf{Spanish} nativo\\
		\textbf{English} proficient
	\end{minipage}
	
\end{document}
